%mechanical study
\newcommand{\x}[1] {\vec{x_#1}}
\newcommand{\y}[1] {\vec{y_#1}}
\newcommand{\z}[1] {\vec{z_#1}}
\newcommand{\base}[1] {
  (\x#1,\y#1,\z#1)
}
\newcommand{\torseur}[2] {%\torseur{force}{solide}
  \left\{
  T_{#1\to #2}
  \right\}
}
\newcommand{\Torseur}[4] {%\Torseur{point}{R}{M}{base}
  \tensor[_{#1}]{
  \left\{
  \begin{array}{l}
    #2 \\
    #3
  \end{array}
  \right\}
  }{_{#4}}
}



\newcommand{\PCGridContour}{ %Contour grille
	\fill[black, very thick] (0,0) rectangle (8,6);
}

\newcommand{\PCGridInside}{ %Grille
	\foreach \i in {1,...,7} {
		\draw [very thin,gray,dotted] (\i,0) -- (\i,6);
	}
	
	\foreach \i in {1,...,5} {
		\draw [very thin,gray,dotted] (0,\i) -- (8,\i);
	}
}

\newcommand{\PCGridContourNum}{ %Contour grille numéroté
	\foreach \i in {0,...,7} {
		\draw (\i+0.5,6) node[above]{$\i$};
	}
	\foreach \i in {0,...,5} {
		\draw (0,5.5-\i) node[left]{$\i$};
	}
}

\newcommand{\PCGridInsideNum}{ %Intérieur grille numéroté
	\foreach \i in {0,...,7} {
		\foreach \j in {0,...,5} {
			\draw (\i+0.5,5.5-\j) node[white]{$(\i,\j)$};
		}
	}
}

\newcommand{\PCGridAxis}{ %Axes
	\fill[rounded corners, color=cyan!20] (8.5,3.7) rectangle (11.5,5.4);
	\draw (10,5.4) node[above] {Axes};
	
	\fill[color=black] (9.5,5) circle [radius=2pt];
	\draw[->] (9.5,5) -- ++(0,-1) node[right] {$\vec{y}$};
	\draw[->] (9.5,5) -- ++(1,0) node[below] {$\vec{x}$};
}

\newcommand{\PCGridDirection}{ %Direction
	\fill[rounded corners, color=cyan!20] (8.5,1) rectangle (11.5,3);
	\draw (10,3) node[above] {Directions};
	\fill[color=black] (10,2) circle [radius=2pt];
	\draw[->] (10,2) -- ++(0.5,0) node[right] {$up$};
	\draw[->] (10,2) -- ++(-0.5,0) node[left] {$down$};
	\draw[->] (10,2) -- ++(0,0.5) node[above] {$left$};
	\draw[->] (10,2) -- ++(0,-0.5) node[below] {$right$};
}

\newcommand{\PCGridFill}[3]{ %Colorie une case
	\fill[color=#3,opacity=0.4] (#1,5-#2) rectangle ++(1,1);
}

\newcommand{\PCGridWall}[2]{ %Construit un mur
	\draw[rounded corners, thick, color=blue] (#1+0.05,5.05-#2) rectangle ++(0.9,0.9);
	\fill[rounded corners, color=blue] (#1+0.1,5.1-#2) rectangle ++(0.8,0.8);
}

\newcommand{\PCPacMan}[3]{ %PacMan
	\fill[color=yellow!90!black, rotate around={#3:(#1+0.5,5.5-#2)}] (#1+0.5,5.5-#2) -- ++(0.283,0.283) arc (45:315:0.4) -- cycle ;	
}

%\newcommand{\PCFruit}[3]{ %Fruit
%	\fill[color=#3] (#1+0.5,5.5-#2) circle [radius=0.2];	
%}

\newcommand{\PCFruit}[3]{
	\foreach \rot in {0,...,5} {
		\fill[color=#3, rotate around={\rot*60:(#1+0.5,5.5-#2)}] (#1+0.68,5.5-#2) circle [radius=0.18];
	}
}

\newcommand{\PCGridUn}{
	\PCGridWall{0}{3}
	\PCGridWall{0}{4}
	\PCGridWall{1}{0}
	\PCGridWall{1}{1}
	\PCGridWall{2}{1}
	\PCGridWall{2}{3}
	\PCGridWall{2}{4}
	\PCGridWall{3}{1}
	\PCGridWall{3}{3}
	\PCGridWall{4}{5}
	\PCGridWall{5}{0}
	\PCGridWall{5}{5}
	\PCGridWall{6}{0}
	\PCGridWall{6}{2}
	\PCGridWall{7}{0}
	\PCGridWall{7}{2}
	\PCGridWall{7}{3}
	\PCGridWall{7}{4}
	\PCGridWall{7}{5}
	\PCFruit{5}{2}{red}
}
