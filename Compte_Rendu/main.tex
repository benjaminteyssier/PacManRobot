\documentclass{article}

\usepackage[utf8]{inputenc}

% Language
\usepackage[french]{babel} %english

% Geometry
\usepackage[includehead, includefoot, margin=1.5cm]{geometry}
\usepackage{multicol}

% FLoat
\usepackage{float}

% Date format
\usepackage[yyyymmdd]{datetime}

% Hyperref
\usepackage[pdfborder={0 0 0}]{hyperref}

% Header and footer
\usepackage{fancyhdr}
\usepackage{lastpage}

% Fonts
%\usepackage{DejaVuSans}
%\usepackage{DejaVuSansMono}
%\usepackage{arev}
\usepackage{amsfonts}
\usepackage{amsmath}
\usepackage{amssymb}

% Colors
\usepackage[table]{xcolor}

% Illustrations
\usepackage{pdfpages}
\usepackage{graphicx}
\usepackage{wrapfig}
\usepackage{graphbox}
\usepackage{tikz}
\usepackage[fixed]{fontawesome5}

% Code listings
\usepackage[breakable]{tcolorbox}
\usepackage{listings}

% Tables% Captions
\usepackage{tabularx}
\usepackage{arydshln}
\usepackage{multirow}

% Lists
\usepackage{enumitem}

% Bibliography
\usepackage{csquotes}
\usepackage[backend=biber, style=numeric, sorting=none]{biblatex}

% Glossaries
\usepackage[acronym]{glossaries}
%, toc, nogroupskip, numberedsection, nonumberlist, section=section

% Captions
\usepackage[hypcap=false]{caption}

% Text
\usepackage{mfirstuc}
\usepackage[squaren,Gray]{SIunits}

% Math
\usepackage{tensor}
% Date format
\renewcommand{\dateseparator}{--}

% Header and footer
\pagestyle{fancy}
\fancyfoot[L]{\today}
\fancyfoot[C]{}
\fancyfoot[R]{Page \thepage \ of \pageref*{LastPage}}

% Fonts
\renewcommand{\familydefault}{\sfdefault}

% Colors
\definecolor{Blue}{RGB}{031, 119, 180}
\definecolor{Orange}{RGB}{255, 127, 14}
\definecolor{Green}{RGB}{044, 160, 44}
\definecolor{Red}{RGB}{214, 039, 040}
\definecolor{Grey}{RGB}{176, 176, 176}
\definecolor{Mines}{RGB}{079, 054, 154}

% Code listings
\tcbset{colback=white, coltext=black, colframe=Mines, boxrule=0.06cm}
\lstset{basicstyle=\normalsize\ttfamily,
        keywordstyle=\color{Orange},
        stringstyle=\color{Green},
        identifierstyle=\color{black},
        commentstyle=\color{Grey},
        showstringspaces=false}

% Lists
%\setlength{\parindent}{0pt}
%\setlist[itemize]{leftmargin=*}
\setitemize[1]{label=\( \bullet \)}
\setitemize[0]{label=\( \circ \)}

% Bibliography
\addbibresource{input/references.bib}

% Glossaries
\makeglossaries
%\makenoidxglossaries

% Maths
\everymath{\displaystyle}

%no need to be alphabetically sorted

%Examples
\newglossaryentry{LED}
{
  name={LED},
  description={light emitting diode},
  first={light emitting diode (LED)}
}

\newglossaryentry{photopol}
{
  name={photopolymérisation},
  description={Procédé de fabrication}
}



%mechanical study
\newcommand{\x}[1] {\vec{x_#1}}
\newcommand{\y}[1] {\vec{y_#1}}
\newcommand{\z}[1] {\vec{z_#1}}
\newcommand{\base}[1] {
  (\x#1,\y#1,\z#1)
}
\newcommand{\torseur}[2] {%\torseur{force}{solide}
  \left\{
  T_{#1\to #2}
  \right\}
}
\newcommand{\Torseur}[4] {%\Torseur{point}{R}{M}{base}
  \tensor[_{#1}]{
  \left\{
  \begin{array}{l}
    #2 \\
    #3
  \end{array}
  \right\}
  }{_{#4}}
}



\newcommand{\PCGridContour}{ %Contour grille
	\fill[black, very thick] (0,0) rectangle (8,6);
}

\newcommand{\PCGridInside}{ %Grille
	\foreach \i in {1,...,7} {
		\draw [very thin,gray,dotted] (\i,0) -- (\i,6);
	}
	
	\foreach \i in {1,...,5} {
		\draw [very thin,gray,dotted] (0,\i) -- (8,\i);
	}
}

\newcommand{\PCGridContourNum}{ %Contour grille numéroté
	\foreach \i in {0,...,7} {
		\draw (\i+0.5,6) node[above]{$\i$};
	}
	\foreach \i in {0,...,5} {
		\draw (0,5.5-\i) node[left]{$\i$};
	}
}

\newcommand{\PCGridInsideNum}{ %Intérieur grille numéroté
	\foreach \i in {0,...,7} {
		\foreach \j in {0,...,5} {
			\draw (\i+0.5,5.5-\j) node[white]{$(\i,\j)$};
		}
	}
}

\newcommand{\PCGridAxis}{ %Axes
	\fill[rounded corners, color=cyan!20] (8.5,3.7) rectangle (11.5,5.4);
	\draw (10,5.4) node[above] {Axes};
	
	\fill[color=black] (9.5,5) circle [radius=2pt];
	\draw[->] (9.5,5) -- ++(0,-1) node[right] {$\vec{y}$};
	\draw[->] (9.5,5) -- ++(1,0) node[below] {$\vec{x}$};
}

\newcommand{\PCGridDirection}{ %Direction
	\fill[rounded corners, color=cyan!20] (8.5,1) rectangle (11.5,3);
	\draw (10,3) node[above] {Directions};
	\fill[color=black] (10,2) circle [radius=2pt];
	\draw[->] (10,2) -- ++(0.5,0) node[right] {$up$};
	\draw[->] (10,2) -- ++(-0.5,0) node[left] {$down$};
	\draw[->] (10,2) -- ++(0,0.5) node[above] {$left$};
	\draw[->] (10,2) -- ++(0,-0.5) node[below] {$right$};
}

\newcommand{\PCGridFill}[3]{ %Colorie une case
	\fill[color=#3,opacity=0.4] (#1,5-#2) rectangle ++(1,1);
}

\newcommand{\PCGridWall}[2]{ %Construit un mur
	\draw[rounded corners, thick, color=blue] (#1+0.05,5.05-#2) rectangle ++(0.9,0.9);
	\fill[rounded corners, color=blue] (#1+0.1,5.1-#2) rectangle ++(0.8,0.8);
}

\newcommand{\PCPacMan}[3]{ %PacMan
	\fill[color=yellow!90!black, rotate around={#3:(#1+0.5,5.5-#2)}] (#1+0.5,5.5-#2) -- ++(0.283,0.283) arc (45:315:0.4) -- cycle ;	
}

%\newcommand{\PCFruit}[3]{ %Fruit
%	\fill[color=#3] (#1+0.5,5.5-#2) circle [radius=0.2];	
%}

\newcommand{\PCFruit}[3]{
	\foreach \rot in {0,...,5} {
		\fill[color=#3, rotate around={\rot*60:(#1+0.5,5.5-#2)}] (#1+0.68,5.5-#2) circle [radius=0.18];
	}
}




\begin{document}

\begin{titlepage}
	\begin{center}
		
		\textsc{\LARGE École des Mines de Saint Étienne}\\[2cm]
		
		\textsc{\Large Défi : Intelligence Artificielle}\\[1cm]
		\textsc{\Large UP3 - AI Practice and Technos: Simulation}\\[2cm]
		
		% Title
		{ \huge \bfseries Compte rendu du projet de simulation}
		\\[1cm]
		\includegraphics[width=0.4\textwidth]{image/logo_mines.png}
		\\[2cm]
		
		% Author and supervisor
		\begin{minipage}{0.8\textwidth}
			\begin{flushleft} \large
				Kalomé \textsc{Botowamungu}\\[0.2cm]
				Nicolas \textsc{Dunou}\\[0.2cm]
				Étienne-Théodore \textsc{Prin}\\[0.2cm]
				Benjamin \textsc{Tessier}\\[0.2cm]
				Timothé \textsc{Tournier}\\
			\end{flushleft}
		\end{minipage}
		
		\vfill
		
	\end{center}
\end{titlepage}

\tableofcontents \label{contents}

%\newpage

\section{Situation et solution proposée}

\section{Évaluation de notre solution}
Pour évaluer la performance de nos robots, le paramètre qui semble le plus pertinent est le nombre de pas de temps nécessaires pour que tous les fruits soient mangés par les pacmans.
Ainsi, deux études sont possibles:
-l’influence de la taille du plateau sur le temps nécessaire aux pacmans pour manger tous les fruits
-l’influence du nombre de pacmans sur le temps nécessaire aux pacmans pour manger tous les fruits
Comme le nombre de fruit est égal au nombre de pacman, le résultat n'est pas complètement analysable car la carte place les objectifs et les pacmans de façon pseudo-aléatoire. Cependant on peut noter que toute solution diminuant le nombre de pas en comparaison au cas où les pacmans n'ont aucune notion de répartition des objectifs est une amélioration du système. Notre indicateur de performance du système pour chaque situation est donc le nombre de pas nécessaires. Si nous souhaitons implémenter de nouvelles méthodes de répartition où de choix des trajectoires, il faudra juger leur performance en fonction de la moyenne du nombre de pas nécessaire dans les différentes configurations de l'environnement (nombre de pac-mans, taille de la carte). L'implémentation de ces différents algorithmes nous ayant déjà demandé un temps de travail très important nous n'avons pas pu réaliser la version basique de choix au hasard ou du choix le plus proche. Cependant, la solution que nous implémentons est obligatoirement plus efficace en terme de nombre de coups puisqu'elle teste toutes les répartitions dont celle qui ressortiraient des méthodes de choix précédentes.



\section{Résultat}

Dans la première étude, on fait varier le nombre de colonnes du plateau en conservant un nombre constant de lignes (20) et de pac-mans (2). Pour chaque taille de plateau donnée, on récupère le nombre de pas de temps pour 6 \textit{seeds} différents afin de faire une moyenne. On remarque qu’en augmentant, la taille du plateau a tendance à faire augmenter le temps nécessaire aux pac-mans pour manger tous les fruits. Cela pouvait être plutôt prévisible car en augmentant la taille du plateau les fruits ont de fortes chances de se retrouver plus loin des pacmans à l’initialisation.
En augmentant le nombre de tests avec des seeds différents et des tailles encore plus grandes on aurait pu avoir une courbe de tendance plus précise qui pourrait peut-être nous donner la relation entre la taille du plateau et le temps de résolution.

\begin{figure}[H]
	\centering
	\includegraphics[width=0.8\textwidth]{image/resultat1}
	\caption{Nombre de pas de temps en fonction de la taille du plateau}
\end{figure}

Pour la seconde étude, on fait varier le nombre de pacmans en conservant une taille constante pour le plateau, 20\times$20. Cette taille a été choisie car elle permet de ne pas avoir un temps de calcul trop long au-delà de 4 robots. 5 plateaux différents ont été testés en faisant varier le nombre de pacmans. On ne peut pas voir directement de lien entre le temps mis par les pacmans pour manger tous les fruits et le nombre de pacmans. Cela est certainement dû au fait que le nombre de fruits est le même que le nombre de pacmans et qu'ils ne sont pas placés de façon aléatoire, donc leur nombre augmentent en même temps dans les tests. Ici aussi, des tests avec un plus grand nombre de plateaux et avec des nombres de pacmans plus grands auraient potentiellement pû permettre d’établir un lien entre les deux données mises en jeu. Une amélioration de notre système pourrait être dans le futur de décorrelé le nombre d'objectifs et le nombre d'agents
\begin{figure}[H]
	\centering
	\includegraphics[width=0.6\textwidth]{image/resultat2}
	\caption{Nombre de pas de temps en fonction du nombre de robots}
\end{figure}


\section{Choix d'implémentation}

\subsection{Obtention d'itinéraire}

\subsection{Obtention de la répartition des objectifs}
Pour obtenir la répartition des objectifs, nous créons un canal mqtt pour que les robots communiquent sur les objectifs qui ne sont pas encore attribués. Ensuite chaque robot prend la liste des robots et des objectifs restants et il calcule la longueur totale de chaque répartition grâce à l'algorithme A^\star$ et choisis l'objectif qui diminue le plus possible la longueur de cette répartition. La longueur de la répartition correspond en fait au maximum de pas qu'un robot a à effectuer avant que tous les objectifs soient remplis Cette implémentation est évidemment non-optimale puisqu'elle ne prend pas en compte la carte connue de chaque robot mais seulement celle du robot qui effectue le calcul.

\subsection{Création de la carte adaptée}


\subsection{Test Grid PacMan}

\noindent
\begin{figure}[!h]
\begin{tikzpicture}[scale=1.6]
	\PCGridContour
	\PCGridContourNum
	\PCGridInside
	\PCGridAxis
	\PCGridDirection
	
	\PCGridWall{3}{3}
	%\PCGridFill{0}{0}{red}
	
	
	\PCPacMan{0}{5}{180}
	\PCFruit{2}{3}{red}
	
	\PCGridInsideNum
\end{tikzpicture}
\caption{Texte de description}
\label{fig:figureExemple}
\end{figure}

\noindent
\begin{figure}[!h]
\begin{tikzpicture}[scale=1.6]
	\PCGridContour
	\PCGridContourNum
	\PCGridInside
	\PCGridAxis
	\PCGridDirection
	
	\PCGridUn
	
	\PCPacMan{3}{2}{0}
	\PCGridInsideNum
\end{tikzpicture}
\caption{Texte de description}
\label{fig:autreFigureExemple}
\end{figure}

Texte d'exemple pour expliquer les Figures \ref{fig:figureExemple} et \ref{fig:autreFigureExemple}, qui sont des grilles.

%\section{Glossaire}
%\printglossary[type=\acronymtype]

%\printglossary

\end{document}