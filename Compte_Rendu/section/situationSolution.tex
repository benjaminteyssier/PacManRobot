\section{Situation et solution proposée}
\subsection{Situations}
Nous avons adapté le jeu \Gls{pacman} pour la simulation multi-agents d'un système complexe. Les agents sont les \Glspl{pacman} qui ont pour objectif de récupérer les \glspl{fruit}, appelés \glspl{fruit} par la suite à travers le labyrinthe. Un exemple est la figure \ref{fig:exemple1}\\
\begin{figure}[!h]
	\begin{tikzpicture}[scale=1.6]
		\PCGridContour
		\PCGridInside
		\PCGridUn
		\PCPacMan{3}{2}{0}
		\PCGridLegend
	\end{tikzpicture}
	\caption{Exemple du jeu}
	\label{fig:exemple1}
\end{figure}

Chaque situation va être unique: les murs sont générés procéduralement selon une \textit{seed}. Les seuls paramètres vont être la \textit{seed}, la taille de la grille et le nombre de \Glspl{pacman}, qui sera aussi le nombre de \glspl{fruit}.

Les \Glspl{pacman} ont chacun en mémoire la carte de ce qu'ils ont visités, ils ne se la partagent pas.

\subsection{Solution proposée}
Chaque \Gls{pacman} va, en fonction de la carte qu'il connaît, calculer le coût de chaque répartition pour en sélectionner la meilleure, et donc son objectif. Pour cela, nous avons implémenté $A^\star$ pour faire le \textit{pathfinding} avec la carte connue.
Ce calcul étant lourd, il n'est fait qu'à l'initialisation et lorsqu'un \Gls{pacman} atteint un \gls{fruit}.

Ensuite, chaque \gls{pacman} va suivre son chemin jusqu'à ce qu'il rencontre un obstacle, auquel cas il recalculeras seulement route à l'aide du \textit{pathfinding}.